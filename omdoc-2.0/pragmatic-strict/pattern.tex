\svnInfo $Id: pattern.tex 9195 2012-04-20 12:26:50Z frabe $
\svnKeyword $HeadURL: https://svn.omdoc.org/repos/omdoc/projects/omdoc-2.0/pragmatic-strict/pattern.tex $

We will now define our extension language (EL). It provides a syntactic means to define pragmatic language features and their semantics in terms of strict {\omdoc}.

\paragraph{Syntax}
EL adds two primitive declarations to {\mmt} theories: \emph{extension declarations} and \emph{pragmatic declarations}:

\begin{center}
\begin{tabular}{|@{\hspace{.4em}}l@{\hspace{.em}}l@{\hspace{.4em}}l@{\hspace{.4em}}|}
\hline
$\Sigma$ &$\bnfas$   &$\Sigma,\,\lfkw{extension}\;\,e = \Phi$\\
         &$\bnfalts$ &$\Sigma,\,\lfkw{pragmatic}\;\,c:\phi$\\
\hline
\end{tabular}
\end{center}

Extension declarations $\lfkw{extension}\;\,e = \Phi$ introduce a new declaration schema $e$ that is described by $\Phi$.
Intuitively, $\Phi$ is a function that takes some arguments and returns a list of declarations, which define the strict semantics of the declaration scheme.

Pragmatic declarations $\lfkw{pragmatic}\;\,c : \phi$ introduce new declarations that make use of a previously declared extension.
Intuitively, $\phi$ applies an extension $e$ a sequence of arguments and evaluates to the returned list of declarations. Thus, $c:\phi$ serves as a pragmatic abbreviation of a list of strict declarations.

The key notion in both cases is that of \emph{theory families}.  They represent
collections of theories by specifying their common syntactic shape.  Intuitively, theory
families arise by putting a $\lambda$-calculus on top of theory fragments $\Sigma$:

\begin{center}
\begin{tabular}{|@{\hspace{.4em}}l@{\tb}l@{\hspace{.4em}}l@{\hspace{.4em}}l@{\hspace{.4em}}|}
\hline
Theory Families & $\Phi$ & $\bnfas$& $\sigfr{\sfr}\bnfalts \pbind{x}{E}\Phi$\\
                & $\phi$ & $\bnfas$& $e \bnfalts \pappl{\Phi}{E}$\\
\hline
\end{tabular}
\end{center}

We group theory families into two non-terminal symbols as shown above: $\Phi$ is formed from theory fragments $\sigfr{\Sigma}$ and $\lambda$-abstraction $\pbind{x}{E}\Phi$. And $\phi$ is formed from references to previously declared extension $e$ and applications of parametric theory families to arguments $E$.
This has the advantage that both $\Phi$ and $\phi$ have a very simple shape.


\begin{example}[Extension Declarations]
In Figure~\ref{fig:mmt-ext} we give the theory $\Assertions$, which declares extensions for axiom and theorem declarations. Their semantics is defined in terms of the Curry-Howard representation of strict {\omdoc}.

Both extensions take a logical formula $F:\form$ as a parameter.
The extension $\axiom$ permits pragmatic declarations of the form $c:\axiom\,F$. These abbreviate {\mmt} constant declarations of the form $c:\ded\,F$.

The extension $\thm$ additionally takes a parameter $D:\ded\,F$, which is a proof of $F$. It permits pragmatic declarations of the form $c : \thm\,F\,D$. These abbreviate {\mmt} constant declarations of the form $c:\ded\,F = D$.

\begin{figure}[ht]
\vspace{-1.5em}
\begin{center}
\begin{twelfsig}
\tsig{\Assertions}\\
\tmeta{\Forms}{}\\
\tpattern{\axiom}{\lflam[\form]{F}}{
 \tsfdecl{c}{\ded\,F}\\
}\\
\tpattern{\thm}{\lflam[\form]{F}\lflam[\ded\,F]{D}}{
 \tsfdecl[D]{c}{\ded\,F}\\
}\\
\tsigend
\end{twelfsig}
\end{center}
\vspace{-1em}
\caption{An MMT Theory with Extension Declarations}\label{fig:mmt-ext}
\vspace{-1.5em}
\end{figure}
\end{example}

Any {\mmt} theory may introduce extension declarations. However, pragmatic declarations
are only legal if the extension that is used has been declared in the meta-theory:

\begin{definition}[Legal Extension Declarations]\label{def:ext-legal}
We say that an extension declaration 
$\extension{e}{\pbind{x_1}{E_1}\ldots\pbind{x_n}{E_n}\sigfr{\Sigma}}$
\noindent is \defemph{legal} in an {\mmt} theory $T$, if the declarations
$x_1:E_1,\ldots,x_n:E_n$ and $\Sigma$ are well-formed in $T$.

This includes the case where $\Sigma$ contains pragmatic declarations.
\end{definition}

\begin{definition}[Legal Pragmatic Declarations]\label{def:prag-legal}
We say that a pragmatic declaration 
$\pragmatic{c}{e\,E_1\ldots E_n}$
\noindent is \defemph{legal} in an {\mmt} theory $T$ if there is a declaration
$\extension{e}{\pbind{x_1}{E'_1}\ldots\pbind{x_n}{E'_n}\sigfr{\Sigma}}$
in the meta-theory of $T$ and each $E_i$ has type $E'_i$.
\end{definition}

Here the typing relation is the one provided by the {\mmt} foundation.

\paragraph{Semantics}
Extension declarations do not have a semantics as such because the extension declared in $M$ only govern what pragmatic declarations are legal in $M$-theories. In particular, contrary to the constant declarations in $M$, a model of $M$ does not interpret the extension declarations.

The semantics of pragmatic declarations is given by elaborating them into strict declarations:
\begin{definition}[Pragmatic-to-Strict Translation \ptos]\label{def:ptos}
A legal pragmatic declaration $\pragmatic{c}{e\;E_1\ldots E_n}$ is translated to a list of strict constant declarations 
\[c.d_1 : \gamma(F_1) = \gamma(D_1),\ldots,c.d_m :\gamma(F_m) = \gamma(D_m)\]

\noindent where $\gamma$ substitutes every $x_i$ with $E_i$ and every $d_j$ with $c.d_j$
%\noindent for substitution
%\[\gamma = \sub{x_1}{E_1},\ldots,\sub{x_n}{E_n},c.d_1/d_1,\ldots,c.d_m/d_m\]
if we have 
\[\extension{e}{\pbind{x_1}{E'_{1}}\ldots\pbind{x_n}{E'_{n}}\sigfr{d_1 : F_1 = D_1,\ldots,d_m : F_m = D_m}}\]
\noindent and every expression $E_i$ has type $E'_i$.
%If we have 
%\[\extension{e}{\pbind{x_1}{E'_{1}}\ldots\pbind{x_n}{E'_{n}}\sigfr{d_1 : F_1 = D_1,\ldots,d_n : F_m = D_m}}\]
%\noindent then a legal pragmatic declaration $\pragmatic{c}{e\;E_1\ldots E_n}$ is translated to a list of strict constant declarations 
%\[c.d_1 : \gamma(F_1) = \gamma(D_1),\ldots,c.d_m :\gamma(F_m) = \gamma(D_m)\]
%\noindent for 
%\[\gamma = \sub{x_1}{E_1},\ldots,\sub{x_n}{E_n},c.d_1/d_1,\ldots,c.d_m/d_m\]
%\noindent where every expression $E_i$ has type $E'_i$.
\end{definition}

\begin{example}\label{ex:ptos}
Consider the following {\mmt} theories in Figure~\ref{fig:ptos}: 
$\HOL$ includes the {\mmt} theory $\Forms$ and declares a constant $\ind$ as the type of individuals. It adds the usual logical connectives and quantifiers -- here we only present truth ($\truth$) and the universal quantifier ($\forall$) -- and introduces equality ($\doteq$) on expressions of type $\alpha$. Then it includes $\Assertions$. This gives $\HOL$ access to the extensions $\axiom$ and $\thm$.
%We give a meta-theory which has access to some extensions. 
%Define HOL via Forms and Assertions.

$\Commutativity$ uses $\HOL$ as its meta-theory and declares a constant $\circ$ that takes two individuals as arguments and returns an individual. It adds a pragmatic declaration named $\comm$ that declares the commutativity axiom for $\circ$ using the axiom extension from $\HOL$.

$\Commutativity'$ is obtained by elaborating $\Commutativity$ according to
Definition~\ref{def:ptos}.
\end{example}

\begin{figure}[ht]
%\begin{center}
\begin{minipage}[b]{4cm}%{0.5\linewidth}
\begin{twelfsig}
\tsig{\HOL}\\
\tmeta{\Forms}{}\\
\decl{\ind}{\kity}\\
\decl{\truth}{\form}\\
%\decl{\false}{\form}\\
%\decl{\neg}{\form\to\form}\\
$\vdots$\\
\decl{\forall}{(\alpha\to\form)\to\form}\\
%\decl{\exists}{(\alpha\to\form)\to\form}\\
\decl{\doteq}{\alpha\to\alpha\to\form}\\[.3em]
\tinclude{\Assertions}{}\\
\tsigend
\end{twelfsig}
\vspace{1.4em}
\end{minipage}
\hspace{.2cm}
\begin{minipage}[b]{5cm}%{0.6\linewidth}
\begin{twelfsig}
\tsig{\Commutativity}\\
\tmeta{\HOL}{}\\
\decl{\circ}{\ind\to\ind\to\ind}\\
%\decl{e}{\ind}\\
\pdecl{\comm}{\axiom\,\forall x:\ind.\,\forall y:\ind.\, x\circ y \doteq y\circ x}\\
%\pdecl{\neutr}{\axiom\,\forall x.\,x\circ e \doteq x}\\
%\pdecl{\neutl}{\axiom\,\forall x.\,e\circ x \doteq x}\\
\tsigend
\end{twelfsig}
\begin{twelfsig}
\tsig{\Commutativity'}\\
\tmeta{\HOL}{}\\
\decl{\circ}{\ind\to\ind\to\ind}\\
%\decl{e}{\ind}\\
\decl{\comm.c}{\ded\,\forall x:\ind.\,\forall y:\ind.\,x\circ y \doteq y\circ x}\\
%\pdecl{\neutr}{\axiom\,\forall x.\,x\circ e \doteq x}\\
%\pdecl{\neutl}{\axiom\,\forall x.\,e\circ x \doteq x}\\
\tsigend
\end{twelfsig}
\end{minipage}
%\end{center}
\caption{A {\ptos} Translation Example}\label{fig:ptos}
\end{figure}

%\begin{example}[Pragmatic Declarations]
%\begin{twelfsig}
%\tsig{\Exponential}\\
%\tmeta{\Assertions}{}\\
%%\decl{\deriv}{\term\to\term}\\
%\pdecl{\expo}{\axiom\,\uexists f.\,(\deriv\,f) = f \wedge (f\,0) = 1}\\
%%\tpattern{\axiom}{\lflam[\form]{F}}{
%% \tsfdecl{m}{\ded\,F}\\
%%}\\
%%\tpattern{\thm}{\lflam[\form]{F}\lflam[\ded\,\form]{D}}{
%% \tsfdecl[D]{m}{\ded\,F}\\
%%}\\
%%\tsigend
%\end{twelfsig}
%\end{example}

%%% Local Variables: 
%%% mode: latex
%%% TeX-master: "paper"
%%% End: 

% LocalWords:  hline bnfas bnfalts sigfr bnfalts pbind bnfalts pappl cdot cdecl
% LocalWords:  emph newcommand mathit ded ded noindent fhorozal omdoc mmt lfkw
% LocalWords:  hspace hspace hspace hspace ldots defemph thm thm twelfsig tsig
% LocalWords:  tmeta tpattern lflam tsfdecl tsigend vspace ptos ptos forall
% LocalWords:  doteq circ comm circ linewidth kity vdots tinclude pdecl neutr
% LocalWords:  neutl deriv uexists
