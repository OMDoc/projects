\svnInfo $Id: syntax.tex 9195 2012-04-20 12:26:50Z frabe $
\svnKeyword $HeadURL: https://svn.omdoc.org/repos/omdoc/projects/omdoc-2.0/pragmatic-strict/syntax.tex $

Our definitions from Section~\ref{sec:pattern} permit pragmatic \emph{abstract} syntax,
which is elaborated into strict abstract syntax.  For human-oriented representations, it
is desirable to complement this with similar extensions of pragmatic \emph{concrete}
syntax.  While the pragmatic-to-strict translation at the abstract syntax level is usually
non-trivial and therefore not invertible, the corresponding translation at the concrete
syntax level should be compositional and bidirectional.

\subsection{\protect\omdoc Concrete Syntax for EL Declarations}\label{sec:omdoc-concrete}

First we extend {\omdoc} with concrete syntax that exactly mirrors the abstract syntax
from Section~\ref{sec:pattern}. The declaration $\extension{e}{\lambda x_1:E_1.\,\ldots\lambda x_n:E_n.\,\{\Sigma\}}$
is written as
\begin{lstlisting}[mathescape,morekeywords={extension,parameter}]
<extension name="$e$">
  <parameter name="$x_1$">$\om{E_1}$</parameter>
  $\hspace{1cm}\vdots$
  <parameter name="$x_n$">$\om{E_n}$</parameter>
  <theory>
  $\hspace{.45cm}\om{\Sigma}$
  </theory>
</extension>
\end{lstlisting}
Here we use the box notation $\om{A}$ to gloss the XML representation of an entity $A$ given in abstract syntax.

Similarly, the pragmatic declaration $\pragmatic{c}{e\,E_1\ldots E_n}$ is written as

\begin{lstlisting}[mathescape,morekeywords={pragmatic}]
<pragmatic name="$c$" extension="$\llquote{M}?e$">
  $\om{E_1}\ldots\om{E_n}$
</pragmatic>
\end{lstlisting}
Here $\llquote{M}$ is the meta-theory in which $e$ is declared so that $\llquote{M}?e$ is the {\mmt} URI of the extension.

\begin{example}
  For the implicit definitions discussed in Section~\ref{sec:pattern}, we use the extension
  $\implDef$ from Figure~\ref{fig:impldef}, which we assume has namespace URI \llquote{U}.
  If $\rho$ is a proof of unique existence for an $f$ such that $f'=f\wedge f(0)=1$, then the exponential function is defined in XML by

\begin{lstlisting}[mathescape,label=lst:expdef,morekeywords={pragmatic}]
<pragmatic name="exp" extension="$\mmturi{\llquote{U}}\Description\implDef$">
  $\om{\lambda f.f'=f\wedge f(0)=1}$  $\om{\rho}$
</pragmatic>
\end{lstlisting}
\end{example}

\subsection{Pragmatic Surface Syntax}

\omdoc is mainly a machine-oriented interoperability format, which is not intended for
human consumption. Therefore, the EL-isomorphic syntax introduced is sufficient in
principle -- at least for the formal subset of \omdoc we have discussed so far.

\omdoc is largely written in the form of ``surface languages'' -- domain-specific
languages that can be written effectively and transformed to \omdoc in an automated
process. For the formal subset of \omdoc, we use a \mmt-inspired superset of the
Twelf/LF~\cite{twelf,lf} syntax, and for informal \omdoc we use
\sTeX~\cite{Kohlhase:ulsmf08}, a semantic extension of {\TeX/\LaTeX}. 

For many
purposes like learning the surface language or styling \omdoc documents,
\defemph{pragmatic surface syntax}, i.e., a surface syntax that is closer to the
notational conventions of the respective domain, has great practical advantages.
It is possible to support, i.e., generate and parse, pragmatic surface
syntax by using the macro/scripting framework associated with most representation
formats.

For instance, we can regain the XML syntax familiar from \omdoc1.2
via notation definitions that transform between \lstinline|pragmatic| elements and the
corresponding {\omdoc} 1.2 syntax. For Twelf/LF, we would extend the module system preprocessor, and for Isabelle we would extend the
SML-based syntax/parsing subsystem. We have also extended \sTeX as an example of a
semi-formal surface language. Here we used the
macro facility of {\TeX} as the computational engine. We conjecture that most practical
surface languages for MKM can be extended similarly.

These translations proceed in two step. Firstly, pragmatic surface syntax is translated into our pragmatic {\mmt} syntax. Our language is designed to make this step trivial: in particular, it does not have to look into the parameters used in a pragmatic surface declaration.
Secondly, pragmatic {\mmt} syntax is type-checked and, if desired, translated into strict {\mmt} syntax.
All potentially difficult semantic analysis is part of this second step.
This design makes it very easy for users to introduce their own pragmatic surface syntax.

%%% Local Variables: 
%%% mode: latex
%%% TeX-master: "paper"
%%% End: 

% LocalWords:  omdoc ednote lstlisting mathescape llquote llquote llqn ldots cn
% LocalWords:  llqn mmt frabe newpart emph implDef impldef lst expdef exu-proof
% LocalWords:  footnotemark exp footnotetext expr exunique exu stex hspace lf
% LocalWords:  morekeywords vdots mmturi ulsmf08 defemph
