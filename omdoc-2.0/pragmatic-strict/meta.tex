\svnInfo $Id: meta.tex 9199 2012-04-20 20:19:00Z frabe $
\svnKeyword $HeadURL: https://svn.omdoc.org/repos/omdoc/projects/omdoc-2.0/pragmatic-strict/meta.tex $

Formal mathematical developments can be classified based on whether they follow the
axiomatic or the definitional method. The former is common for logics where theories
declare primitive constants and axioms. The latter is common for foundations of
mathematics where a fixed theory (the foundation) is extended only by defined constants
and theorems.  In {\mmt}, both the logic and the foundation are represented as a
meta-theory $M$, and the main difference is that the definitional method does not permit
undefined constants in $M$-theories.

However, this treatment does not capture conservative extension principles:
These are meta-theorems that establish that certain extensions are acceptable even if they are not definitional.
We can understand them as intermediates between axiomatic and definitional extensions: They may be axiomatic but are essentially as safe as definitional ones. 

To make this argument precise, we use the following definition:

\begin{definition}
We call the theory family
$\Phi=\pbind{x_1}{E'_1}\ldots\pbind{x_n}{E'_n}\sigfr{\Sigma}$
\defemph{conservative} for $M$ if for every $M$-theory $T$ and all $E_1:E'_1,\ldots,E_n:E'_n$, every model of $T$ can be extended to a model of $T,\gamma(\Sigma)$, where $\gamma$ substitutes every $x_i$ with $E_i$.

  An extension declaration $\extension{e}{\Phi}$ is called \defemph{derived} if all
  constant declarations in $\Sigma$ have a definiens; otherwise, it is called \defemph{primitive}.
\end{definition}

Primitive extension declarations correspond to axiom declarations because they postulate that certain extensions of $M$ are legal. The proof that they are indeed conservative is a meta-argument that must be carried out as a part of the proof that $M$ is an adequate {\mmt} representation of the represented formalism.
Similarly, derived extension declarations correspond to theorem declarations because their conservativity follows from that of the primitive ones. More precisely:
If all primitive extension principles in $M$ are conservative, then so are all derived ones.

In the following, we will recover built-in extension statements of common representation formats as special cases of our extension declarations.
We will follow a \emph{little foundations} paradigm and state every extensions in the smallest theory in which it is meaningful. Using the {\mmt} module system, this permits maximal reuse of extension definitions. Moreover, it documents the (often implicit) foundational assumptions of each extension.

\paragraph{Implicit Definitions in {\omdoc}}
Implicit definitions of {\omdoc} 1.2 are captured using the following derived extension declaration.
If the theory $\Description$ in Figure~\ref{fig:impldef} is included into a meta-theory $M$, then $M$-theories may use implicit definitions.

\begin{figure}[ht]
\vspace{-1.5em}
\begin{center}
\begin{twelfsig}
\tsig{\Description}\\
\tmeta{\Forms}{}\\
\decl{\uexists}{(\alpha\to\form)\to\form}\\
\decl{\descr}{(\alpha\to\form)\to\alpha}\\
\decl{\descr_{\ax}}{\ded\,\uexists x\,P\,x\to \ded\,P\,(\descr\,P)}\\
[.5em]
\tpattern{\implDef}{\lflam[\kity]{\alpha}\lflam[\alpha\to\form]{P}\lflam[\ded\,\uexists x:\alpha.\,P\,x]{m}}{
 \tsfdecl{c}{\alpha\;\;=\;\;\descr\,P}\\
 \tsfdecl{c_\ax}{\ded\,\uexists x:\alpha.\,P\,x}\\
}\\
\tsigend
\end{twelfsig}
\end{center}
\vspace{-2em}
\caption{An Extension for Implicit Definitions}\label{fig:impldef}
\vspace{-1em}
\end{figure}

Note that $\Description$ requires two other connectives: A description operator ($\descr$) and a unique existential ($\uexists$) are needed to express the meaning of an implicit definition.
We deliberately assume only those two operators in order to maximize the re-usability of this theory: Using the {\mmt} module system, any logic $M$ in which these two operators are definable can import the theory $\Description$.
%In particular, we do not define the unique existential quantifier in terms of equality so that $\Description$ can be reused if $M$ does not have to have equality.

More specifically, $\Description$ introduces the definite description operator as a new
binding operator ($\descr$), and describes its meaning by the axiom $\uexists
x\,P(x)\impl\,P\,(\descr\,P)$ formulated in $\descr_{\ax}$ for any predicate $P$ on $\alpha$.
The extension $\implDef$ permits pragmatic declarations of the form $f:
\implDef\,\alpha\,P\,m$, which defines $f$ as the unique object which makes the property
$P$ valid. This leads to the well-defined condition that there is indeed such a unique
object, which is discharged by the proof $m$.  The pragmatic-to-strict translation from
Section~\ref{sec:pattern} translates the pragmatic declaration $f: \implDef\,\alpha\,P\,m$
to the strict constant declarations $f.c : \alpha = \descr\,P$ and $f.c_{\ax} :
\ded\,\uexists x:\alpha\,P\,x$.



\paragraph{Mizar-Style Functor Definitions}
The Mizar language \cite{mizar} provides a wide (but fixed) variety of special statements, most of which can be understood as conservative extension principles for first-order logic. A comprehensive list of the corresponding extension declarations can be found in \cite{IKR:mizar:11}.
We will only consider one example in Figure~\ref{fig:mizar}.


\begin{figure}[ht]
\vspace{-2em}
\begin{center}
\begin{twelfsig}
\tsig{\FunctorDefinitions}\\
\tmeta{\Forms}{}\\
\decl{\wedge}{\form\to\form\to\form}\\
\decl{\impl}{\form\to\form\to\form}\\
\decl{\forall}{(\alpha\to\form)\to\alpha}\\
\decl{\exists}{(\alpha\to\form)\to\alpha}\\
\decl{\eqn}{\alpha\to\alpha\to\form}\\
%\decl{\means}{(\alpha\to\form)\to\form}\\[.5em]
%\tpattern{\functor}{\lflam[\kity]{\alpha}\lflam[\kity]{\beta}\lflam[\alpha\to\beta\to\form]{\means}}{
% \tsfdecl{f}{\alpha\to\beta}\\
% \tsfdecl{\defthm}{\means\,x\,(f\,x)}\\
%}\\
\multicolumn{4}{@{\tb}l}{$\lfkw{extension}\;\functor\;=$}\\
& &$\lflam[\kity]{\alpha}\lflam[\kity]{\beta}\lflam[\alpha\to\beta\to\form]{\means}$\\
& &$\lflam[\ded\,\forall x:\alpha.\,\exists y:\beta.\,\means\,x\,y]{\existence}$\\
& &$\lflam[\ded\,\forall x:\alpha.\,\forall y:\beta.\,\forall y':\beta.\,\means\,x\,y\wedge \means\,x\,y'\impl y\eqn y']{\uniqueness}\,\{$\\
\multicolumn{4}{@{\tb\tb}l}{
\begin{tabular}{lll}
$\tb f$ &$:$ &$\alpha\to\beta$\\
$\tb\defthm$ & $:$ &$\ded\,\forall x:\alpha.\,\means\,x\,(f\,x)$\\
\end{tabular}}\\
&& $\}$\\
\tsigend
\end{twelfsig}
\end{center}
\vspace{-2.5em}
\caption{An Extension for Mizar-Style Functor Definitions}\label{fig:mizar}
\vspace{-1.5em}
\end{figure}

The theory $\FunctorDefinitions$ describes Mizar-style implicit definition of a unary function symbol (called a \emph{functor} in Mizar). This is different from the one above because it uses a primitive extension declaration that is well-known to be conservative. In Mizar, the axiom $\defthm$ is called the definitional theorem induced by the implicit definition. Using the extension $\functor$, one can introduce pragmatic declarations of the form $\pragmatic{c}{\functor\,A\,B\,P\,E\,U}$ that declare functors $c$ from $A$ to $B$ that are defined by the property $P$ where $E$ and $U$ discharge the induced proof obligations.

\paragraph{Flexary Extensions}
The above two examples become substantially more powerful if they are extended to implicit definitions of functions of arbitrary arity.
This is supported by our extension language by using an LF-based logical framework with term sequences and type sequences.
We omit the formal details of this framework here for simplicity and refer to \cite{Hor:patterns} instead.
We only give one example in Figure~\ref{fig:casebased} that demonstrates the potential.

\begin{figure}[ht]
\vspace{-1.5em}
\begin{center}
\begin{twelfsig}
\tsig{\CaseBased}\\
\tmeta{\Forms}{}\\
%$\wedge$\\
%$\impl$\\
%$\forall$\\
\decl{\wedge}{\form^n\to\form}\\
\decl{\vee^{!}}{\form^n\to\form}\\
\decl{\impl}{\form\to\form\to\form}\\
\decl{\forall}{(\alpha\to\form)\to\form}\\
\multicolumn{4}{@{\tb}l}{$\lfkw{extension}\;\casedef\;=\;\lflam[\mathbb{N}]{n}\lflam[\kity]{\alpha}\lflam[\kity]{\beta}\lflam[(\alpha\to\form)^n]{c}$}\\
& &$\hspace{2.65cm}\lflam[(\alpha\to\beta)^n]{d}\lflam[\ded\,\forall x:\alpha.\, \vee^{!} \lfseqbind{c_i\,x}{i}{n}]{\rho}\{$ \\
\multicolumn{4}{@{\tb\tb}l}{
\begin{tabular}{lll}
$f$ &$:$ &$\alpha\to\beta$\\
$\ax$ & $:$ &$\ded\,\forall x:\alpha.\,\wedge\,\lfseqbind{c_i\,x\,\impl\,(f\,x) = (d_i\,x)}{i}{n}$\\
\end{tabular}}\\
\multicolumn{4}{@{\tb}l}{$\}$}\\
%\tpattern{\caseBasedDef}{\lflam[\mathbb{N}]{n}\lflam[\kity]{\alpha}\lflam[\kity]{\beta}\lflam[(\alpha\to\form)^n]{c}\lflam[(\alpha\to\beta)^n]{d}\lflam[\ded\,\forall x:\alpha.\, \vee^{!} \lfseqbind{c_i\,x}{i}{n}]{m}}{
% \tsfdecl{f}{\alpha\to\beta}\\
% \tsfdecl{\ax}{\ded\,\forall x:\alpha.\,\wedge\,\lfseqbind{c_i\,x\,\impl\,(f\,x) = (d_i\,x)}{i}{n}}\\ 
%}\\
\tsigend
\end{twelfsig}
\end{center}
\vspace{-2em}
\caption{An Extension for Case-Based Definitions}\label{fig:casebased}
\vspace{-1.5em}
\end{figure}

The theory $\CaseBased$ introduces an extension that describes the case-based definition of a unary function $f$ from $\alpha$ to $\beta$ that is defined using $n$ different cases where each case is guarded by the predicate $c_i$ together with the respective definiens $d_i$.
Such a definition is well-defined if for all $x \in\alpha$ exactly one out of the $c_i\,x$ is true.
Note that these declarations use a special sequence constructor: for example, $\lfseqbind{c_i\,x}{i}{n}$ simplifies to the sequence $c_1\,x\,,\ldots,c_n\,x$.
Moreover, $\wedge$ and $\vee^!$ are flexary connectives, i.e., they take a flexible number of arguments.
In particular, $\vee^!(F_1,\ldots,F_n)$ holds if exactly one of its arguments holds.

The pragmatic declaration $\pragmatic{f}{\casedef\,n\,\alpha\,\beta\,c_1\ldots c_n\,d_1\ldots d_n\,\rho}$
%corresponds to the function definition $f(x) =\left\{
%\begin{array}{lll}
%  d_1(x) & \quad\textrm{if}\quad & c_1(x) \\
% \vdots && \vdots\\
% d_n(x) & \quad\textrm{if}\quad & c_n(x)
%\end{array}\right.
%$
corresponds to the following function definition: \[f(x) =\left\{
\begin{array}{lll}
  d_1(x) & \quad\textrm{if}\quad & c_1(x) \\
 \vdots && \vdots\\
 d_n(x) & \quad\textrm{if}\quad & c_n(x)
\end{array}\right.
\]


%In the extension $\casedef$, the unary function $f$ is defined using $n$ different cases guarded by the predicates $c_i$ together with the respective definientia $d_i$. 
%Here, $(\alpha\to\form)^$ denotes the sequence of types $\alpha\to\form$ of length $n$.


%\multirow{3}{*}{Immediate} & RR & Round Robin \\
%\[
%f(x) =
%\begin{cases}
% D_1, & \text{if } C_1 \text{ is even} \\
% D_2, & \text{if } C_2 \text{ is odd}
%\end{cases}
%\]


\paragraph{HOL-Style Type Definitions}
Due to the presence of $\lambda$-abstraction and a description operator in HOL \cite{churchtypes}, a lot of common extension principles become derivable in HOL, in particular, implicit definitions.

But there is one primitive definition principle that is commonly accepted in HOL-based formalizations of the definitional method: A Gordon/HOL type definition \cite{hol} introduces a new type that is axiomatized to be isomorphic to a subtype of an existing type. This cannot be expressed as a derivable extension because HOL does not use subtyping.


\begin{figure}[ht]
\vspace{-1.5em}
\begin{center}
\begin{twelfsig}
\tsig{\Types}\\
\tmeta{\Forms}{}\\
\decl{\forall}{(\alpha\to\form)\to\form}\\
\decl{\exists}{(\alpha\to\form)\to\form}\\
\decl{\doteq}{(\alpha\to\alpha)\to\form}\\
\tpattern{\typeDef}{\lflam[\kity]{\alpha}\lflam[\alpha\to\form]{A}\lflam[\ded\,\exists x:\alpha.\,A\,x]{P}}{
 \tsfdecl{T}{\kity}\\
 \tsfdecl{\Rep}{T\to\alpha}\\
 \tsfdecl{\Abs}{\alpha\to T}\\
 \tsfdecl{{\Rep'}}{\ded\,\forall x:T.\,A\,(\Rep\,x)}\\
 \tsfdecl{{\RepInv}}{\ded\,\forall x:T.\,\Abs\,(\Rep\,x) \doteq x}\\
 \tsfdecl{{\AbsInv}}{\ded\,\forall x:\alpha.\,A\,x\impl\Rep\,(\Abs\,x) \doteq x}\\
}\\
\tsigend
\end{twelfsig}
\vspace{-1.5em}
\caption{An Extension for HOL-Style Type Definitions}\label{fig:typedef}
\end{center}
\vspace{-1.5em}
\end{figure}

The theory $\Types$ in Figure~\ref{fig:typedef}, formalizes this extension principle. Our symbol names follow the implementation of this definition principle in Isabelle/HOL \cite{isabellehol}.
%, which takes a type $\alpha$, a predicate $A$ on $\alpha$ and a proof that $A$ is non-empty and  
Pragmatic declarations of the form $\pragmatic{t}{\typeDef\,\alpha\,A\,P}$ introduce a new non-empty type $t$ isomorphic to the predicate $A$ over $\alpha$. Since all HOL-types must be non-empty, a proof $P$ of the non-emptiness of $A$ must be supplied. 
More precisely, it is translated to the following strict constant declarations:
\begin{itemize}
\item $t.T: \kity$ is the new type that is being defined,
\item $t.\Rep: t.T\to\alpha$ is an injection from the new type $t.T$ to $\alpha$,
\item $t.\Abs:\alpha\to t.T$ is the inverse of $t.\Rep$ from $\alpha$ to the new type $t.T$,
\item $t.\Rep'$ states that the property $A$ holds for any term of type $t.T$,
\item $t.\RepInv$ states that the injection of any element of type $t.T$ to $\alpha$ and back is equal to itself,
\item $t.\AbsInv$ states that if an element satisfies $A$, then injecting it to $t.T$ and back is equal to itself.
\end{itemize}

HOL-based proof assistants implement the type definition principle as a built-in statement.
They also often provide further built-in statements for other definition principles that become derivable in the presence of type definitions, e.g., a definition principle for record types.
For example, in Isabelle/HOL \cite{isabellehol}, HOL is formalized in the Pure logic underlying the logical framework Isabelle \cite{isabelle}.
But because the type definition principle is not expressible in Pure, it is implemented as a primitive Isabelle feature that is only active in Isabelle/HOL.

%For all $x$, $f(x)$ is the unique $y$ such that $p(x,f(x))$.

%%% Local Variables: 
%%% mode: latex
%%% TeX-master: "paper"
%%% End: 

% LocalWords:  wrapfigure vspace twelfsig tsig kity ded ineq tpattern lflam thm
% LocalWords:  tsfdecl tsfdecldef tsigend ednote ptos impldef texttt forall lst
% LocalWords:  p2srule lstlisting mathescape expfun omdoc bigl desctheory impl
% LocalWords:  uexists lfseqbind tm tp tpj fhorozal newpart mmt pbind ldots
% LocalWords:  sigfr defemph tmeta descr doteq defthm Flexary casebased lfkw
% LocalWords:  casedef mathbb hspace noindent multirow Bigg textrm vdots RepInv
% LocalWords:  churchtypes AbsInv isabellehol
