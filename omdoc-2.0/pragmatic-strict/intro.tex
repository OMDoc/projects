\svnInfo $Id: intro.tex 9171 2012-03-05 08:39:33Z kohlhase $
\svnKeyword $HeadURL: https://svn.omdoc.org/repos/omdoc/projects/omdoc-2.0/pragmatic-strict/intro.tex $

The development of representation languages for mathematical knowledge is one of the
central concerns of the MKM community. After all, practical mathematical knowledge
management consists in the manipulation of expressions in such languages. To be
successful, MKM representation formats must balance multiple concerns. A format should be
expressive and flexible (for depth and ease of modeling), foundationally unconstrained
(for coverage), regular and minimal (for ease of implementation), and modular and
web-transparent (for scalability). Finally, the format should be elegant, feel natural to
mathematicians, and be easy to read and write. Needless to say that this set of
requirements is over-constrained so that the design problem for MKM representation formats
lies in relaxing some of the constraints to achieve a global optimum.

In languages for formalized mathematics, it is standard practice to define a minimal
core language that is extended by macros, functions, or notations.
For example, Isabelle \cite{isabelle} provides a rich language of notations, abbreviations, syntax and printing translations, and a number of definitional forms.
In narrative formats for mathematics, for
instance, the {\TeX/\LaTeX} format -- arguably the most commonly used format for
representing mathematical knowledge -- goes a similar way, only that the core language is
given by the {\TeX} layout primitives and the translation is realized by macro expansion
and is fully under user control. This extensibility led to the profusion of
user-defined {\LaTeX} document classes and packages that has made {\TeX/\LaTeX} so
successful.

However, the fully unconstrained nature of the extensibility makes ensuring invariants and
machine support very difficult, and thus this approach is not immediately applicable to
content markup formats. There, MathML3~\cite{CarlisleEd:MathML3:base} is a good example of
the state of the art. It specifies a core language called ``strict content MathML'' that
is equivalent to OpenMath~\cite{BusCapCar:2oms04} and ``full content MathML''. The first
subset uses a minimal set of elements representing the meaning of a mathematical
expression in a uniform, regular structure, while the second one tries to strike a
pragmatic balance between verbosity and formality. The meaning of non-strict expressions
is given by a fixed translation: the ``strict content MathML translation'' specified in
section 4.6 of the MathML3 recommendation~\cite{CarlisleEd:MathML3:base}.

This language design has the advantage that only a small, regular sublanguage has to be
given a mathematical meaning, but a larger vocabulary that is more intuitive to
practitioners of the field can be used for actual representation. Moreover, semantic
services like validation only need to be implemented for the strict subset and can be
extended to the pragmatic language by translation. Ultimately, a representation format
might even have multiple pragmatic front-ends geared towards different audiences. These
are semantically interoperable by construction.
\medskip

The work reported in this paper comes from an ongoing language design effort, where we
want to redesign our \omdoc format~\cite{Kohlhase:OMDoc1.2} into a minimal, regular core
language (\emph{strict \omdoc2}) and an extension layer (\emph{pragmatic \omdoc2}) whose
semantics is given by a ``pragmatic-to-strict'' ($\ptos$) translation.  While this problem
is well-understood for mathematical \emph{objects}, extension frameworks \emph{at the
  statement level} seem to be restricted to the non-semantic case, e.g. the
\texttt{amsthm} package for {\LaTeX}.

Languages for mathematics commonly permit a variety of pragmatic statements, e.g.,
implicit or case-based definitions, type definitions, theorems, or proof schemata.  But
representation frameworks for such languages do not include a generic mechanism that
permits introducing arbitrary pragmatic statements --- instead, a fixed set is built into
the format.  Among logical frameworks, Twelf/LF~\cite{twelf,lf} permits two statements:
defined and undefined constants. Isabelle~\cite{isabelle} and Coq~\cite{coq} permit much
larger, but still fixed sets that include, for example, recursive case-based function
definitions.  Content markup formats like \omdoc permit similar fixed sets.

A large set of statements is desirable in a representation format in order to model the
flexibility of individual languages. A large \emph{fixed} set on the other hand is
unsatisfactory because it is difficult to give a theoretical justification for fixing any
specific set of statements.  Moreover, it is often difficult to define the semantics of a
built-in statement in a foundationally unconstrained representation format because many
pragmatic statement are only meaningful under certain foundational assumptions.
%For example, an existential quantifier is needed to express the well-definedness condition of an implicit definition.
\medskip

In this paper we present a general formalism for adding new pragmatic statement forms to
our \omdoc format; we have picked \omdoc for familiarity and foundation-independence; any
other foundational format can be extended similarly. Consider for instance the pragmatic
statement of an ``implicit definition'', which defines a mathematical object by describing
it so accurately, that there is only one object that fits this description. For instance,
the exponential function $exp$ is defined as the (unique) solution of the differential
equation $f=f'$ with $f(0)=1$. This form of definition is extensively used in practical
mathematics, so pragmatic {\omdoc} should offer an infrastructure for it, whereas strict
{\omdoc} only offers ``simple definitions'' of the form $c:=d$, where $c$ is a new symbol
and $d$ any object.  In our extension framework, the $\ptos$ translation provides the
semantics of the implicit definition in terms of the strict definition $exp:=\descr
f.(f'=f\wedge f(0)=1)$, where $\descr$ is a ``definite description operator'': Given an
expression $A$ with free variable $x$, such that there is a unique $x$ that makes $A$
valid, $\descr x.A$ returns that $x$, otherwise $\descr x.A$ is undefined.

Note that the semantics of an implicit definition requires a definite description operator.
While most areas of mathematics at least implicitly assume its existence, it should not be required in general because that would prevent the representation of systems without one.
Therefore, we make these requirements explicit in a special theory that defines the new pragmatic statement and its strict semantics. This theory must be imported in order for implicit definitions to become available.
Using our extension language, we can recover a large number of existing pragmatic statements as definable special cases, including many existing ones of {\omdoc}.
Thus, when representing formal languages in {\omdoc}, authors have full control what pragmatic statements to permit and can define new ones in terms of existing ones.
\medskip

In the next section, we will recap those parts of \omdoc that are
needed in this paper. In Section~\ref{sec:pattern}, 
we define our extension language, and in Section~\ref{sec:meta}, we look at particular extensions that are motivated by mathematical practice. Finally, in Section~\ref{sec:syntax}, we will address the question
of extending the concrete syntax with pragmatic features as well.



%%% Local Variables: 
%%% mode: latex
%%% TeX-master: "paper"
%%% End: 

% LocalWords:  wrapfigure vspace hline ednote emph emph RabKoh prel Rudnicki lf
% LocalWords:  isabelle-isar aomp92 omdoc ptos frabe medskip texttt amsthm exp
% LocalWords:  descr
