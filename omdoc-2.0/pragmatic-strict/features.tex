\svnInfo $Id: features.tex 9179 2012-03-14 20:20:33Z fhorozal $
\svnKeyword $HeadURL: https://svn.omdoc.org/repos/omdoc/projects/omdoc-2.0/pragmatic-strict/features.tex $
\section{Generic Elaboration of Pragmatic Language Features}
 A \defemph{feature} consists of three components:
\begin{itemize}
  \item an {\mmt} theory, called the \defemph{feature theory},
  \item an extension of the {\mmt} grammar and type system with symbol-level declarations, called the \defemph{feature-dependent syntax},
  \item an \defemph{elaboration} of the feature-dependent syntax into {\mmt} syntax
\end{itemize}

We will abbreviate the URI \url{http://cds.omdoc.org/features} with $\cn{Feat}$.

\paragraph{Logic}
The feature of being a logic is defined as follows.
The feature theory is given by
\[\thdecl{\feath{Logic}}{\symdd{o}{}{},\;\symdd{ded}{}{}}\]
The pragmatic syntax is given by
\[Sym \;\bnfas\; \mmtaxiom{a}{F}\]
The well-formedness rule is
\[\ianc{\otermtype{\TG}{\qT}{F}{o}}
       {\osym{\TG}{\qT}{\mmtaxiom{a}{F}}}
       {}\]
\ednote{Alternatively, the semantics can be defined through elaboration. Then this would become a theorem.}

The elaboration is given by
\[\elaborate{\mmtaxiom{a}{F}}
            {\symdd{a}{\oma{ded,F}}}{}\]
\ednote{ded here actually refers to the meaning of ded within $\qT$. This must be
  discussed.}
%%% Local Variables: 
%%% mode: latex
%%% TeX-master: "paper"
%%% End: 

% LocalWords:  defemph mmt cn thdecl feath symdd symdd ded Sym bnfas mmtaxiom
% LocalWords:  ianc otermtype osym ednote
