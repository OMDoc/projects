\svnInfo $Id: mmt.tex 9195 2012-04-20 12:26:50Z frabe $
\svnKeyword $HeadURL: https://svn.omdoc.org/repos/omdoc/projects/omdoc-2.0/pragmatic-strict/mmt.tex $

\begin{wrapfigure}r{3.2cm}\vspace*{-2em}
\begin{tikzpicture}[yscale=.7]
\draw (0,0) rectangle (3,3);
\draw (0,1) -- (2,1);
\draw (0,2) -- (2,2);
\draw (2,0) -- (2,3);
\node (thy) at (1,2.5) {Theories};
\node (st) at (1,1.5) {Statements};
\node (obj) at (1,0.5) {Objects};
\node (doc) at (2.5,1.5){\begin{sideways}Documents\end{sideways}};
\end{tikzpicture}
\vspace*{-2em}
\end{wrapfigure}
\omdoc is a comprehensive content-based format for representing mathematical knowledge and
documents. It represents mathematical knowledge at three levels: mathematical formulae at the
\emph{object level}, symbol declarations, definitions, notation definitions, axioms, theorems, and
proofs at the \emph{statement level}, and finally modular scopes at the \emph{theory level}. Moreover, it adds
an infrastructure for representing functional aspects of \emph{mathematical documents} at the
content markup level. \omdoc1.2 has been successfully used as a representational basis in applications
ranging from theorem prover interfaces, via knowledge based up to eLearning systems. To allow
this diversity of applications, the format has acquired a large, interconnected set of language
constructs motivated by coverage and user familiarity (i.e., by pragmatic concerns) and not by
minimality and orthogonality of language primitives (strict concerns).

To reconcile these language design issues for \omdoc2, we want to separate the format into a \emph{strict} core language
and a \emph{pragmatic} extension layer that is elaborated into strict \omdoc via a ``pragmatic-to-strict'' ($\ptos$) translation.

For strict \omdoc we employ the foundation-independent, syntactically minimal {\mmt} framework
(see below). For pragmatic \omdoc, we aim at a language that is feature-complete with
respect to \omdoc1.2~\cite{Kohlhase:OMDoc1.2}, but incorporates language features from
other MKM formats, most notably from Isabelle/Isar~\cite{isar},
PVS~\cite{pvs}, and Mizar~\cite{mizar}.

The {\mmt} language was emerged from a complete redesign of the formal core\footnote{We are currently
  working on adding an informal (natural language) representation and a non-trivial (strict)
  document level to {\mmt}, their lack does not restrict the results reported in this paper.} of \omdoc
focusing on foundation-independence, scalability, modularity, while maintaining coverage of formal
systems. The {\mmt} language is described in~\cite{RK:mmt:10} and implemented
in~\cite{project:mmt}.

\begin{wrapfigure}{l}{5.2cm}\vspace*{-2em}
\begin{center}
\begin{tikzpicture}[xscale=.95,yscale=.7]
\node[thy] (A) at (0,3)  {$\mathit{LF}$};
\node[thy] (A') at (2,3)  {$\mathit{Isabelle}$};
\node[thy] (C) at (-1,1.5)   {$\mathit{FOL}$};
\node[thy] (C') at (1,1.5) {$\mathit{HOL}$};
\node[thy] (E) at (-2,0) {$\mathit{Monoid}$};
\node[thy] (E') at (0,0)  {$\mathit{Ring}$};
\draw[meta](A) -- (C);
\draw[meta](A) -- (C');
\draw[meta](C) -- (E);
\draw[meta](C) -- (E');
\draw[struct](A) --node[above] {} (A');
\draw[struct](C) --node[above] {} (C');
\draw[struct](E) --node[above] {} (E');
%\draw[struct](A) --node[above] {$m$} (A');
%\draw[struct](C) --node[above] {$m'$} (C');
%\draw[struct](E) --node[above] {$i$} (E');
\end{tikzpicture}
\caption{An {\mmt} Theory Graph}\label{fig:mmt:intro:metatheory}
\end{center}\vspace*{-2em}
\end{wrapfigure}

MMT uses \defemph{theories} as a single primitive to represent formal systems such as logical frameworks, logics, or theories.
These form theory graphs such as the one on the left, where single arrows $\rightarrow$ denote theory translations and hooked arrows $\hookrightarrow$ denote the meta-theory relation between two theories.
The theory $\mathit{FOL}$ for first-order logic is the meta-theory for $\mathit{Monoid}$ and $\mathit{Ring}$. And the theory $\mathit{LF}$
for the logical framework LF~\cite{lf} is the meta-theory of $\mathit{FOL}$ and $\mathit{HOL}$ for higher-order logic.
In general, we describe the theories with meta-theory $M$ as \defemph{$M$-theories}.
The importance of meta-theories in {\mmt} is that the syntax and semantics of $M$ induces the syntax and semantics of all $M$-theories.
For example, if the syntax and semantics are fixed for $\mathit{LF}$, they determine those of $\mathit{FOL}$ and $\mathit{Monoid}$.

At the statement level, {\mmt} uses \defemph{constant} declarations as a single primitive to represent all OMDoc statement declarations. These are differentiated by the type system of the respective meta-theory. In particular, the Curry-Howard correspondence is used to represent axioms and theorems as plain constants (with special types).
\medskip

In Figure~\ref{fig:mmt-grammar}, we show a small fragment of the {\mmt} grammar that we need in the remainder of this paper. Meta-symbols of the BNF format are given in \bnf{color}.
%the salient language features layered according the \omdoc stratification.

%{\mmt} is meant to be applicable to all base languages based on \emph{theories}. Relations between theories (theory morphisms) are represented as {\mmt} \emph{views}. Both theories and views form the {\mmt} \emph{theory} level. A \emph{theory morphism} $\overline{\sigma} : S \rightarrow T$ is a \emph{signature morphism} $\sigma : S \rightarrow T$ interpreting all symbols of $S$ in $T$ and, in addition, $\overline{\sigma}$ translates all theorems of $S$ to theorems of $T$. {\mmt} uses the \emph{Curry-Howard representation} to drop the distinction between symbols and axioms (and thus between signatures and theories). As a result, {\mmt} needs only theories and theory morphisms.

\begin{figure}[ht]
\begin{center}
\begin{tabular}{|@{\hspace{.4em}}l@{\tb}l@{\hspace{.4em}}l@{\hspace{.4em}}l@{\hspace{.4em}}|}
\hline
Modules              & $G$      &$\bnfas$& $\bnf{(}\lfkw{theory}\;\,T = \{\Sigma\}\bnf{)^{\ast}}$\\
Theories             & $\Sigma$ &$\bnfas$& $\cdot\bnfalts\Sigma,\,%\lfkw{constant}\;\,
c \bnf{[}:E \bnf{][}=E\bnf{]}\bnfalts \lfkw{meta}\;\,T$\\   
Contexts     & $\Gamma$ &$\bnfas$& $\cdot\bnfalts\Gamma,\,x\bnf{[}:E\bnf{]}$\\ 
Expressions          & $E$      &$\bnfas$& $x\bnfalts c\bnfalts E\,E^{\bnf{+}}\bnfalts E\,\Gamma.\,E$ \\
\hline
\end{tabular}
\end{center}
\caption{{\mmt} Grammar}\label{fig:mmt-grammar}
\end{figure}

The module level of {\mmt} introduces \emph{theory declarations} $\lfkw{theory}\;\,T = \{\Sigma\}$. %\stackrel{M}{=}
Theories $\Sigma$ contain \emph{constant declarations} $c[:E_1][=E_2]$ that introduce named atomic expressions $c$ with optional type $E_1$ or definition $E_2$. Moreover, %theories can include other theories $T$ via $\lfkw{include}\;T$, and 
each theory may declare its meta-theory $T$ via $\lfkw{meta}\;T$.

{\mmt} expressions are a fragment of OpenMath~\cite{openmath} objects, for which we introduce a short syntax. They are formed from variables $x$, constants $c$, applications $E\,E_1\,\ldots\, E_n$ of functions $E$ to a sequence of arguments $E_i$, and bindings $E_1\,\Gamma.\,E_2$ that use a binder $E_1$, a context $\Gamma$ of bound variables, and a scope $E_2$.
Contexts $\Gamma$ consist of variables $x[:E]$ that can optionally attribute a type $E$.

The semantics of {\mmt} is given in terms of \emph{foundations} for the upper-most meta-theories. Foundations define in particular the typing relation between expressions, in which {\mmt} is parametric.
For example, the foundation for $\mathit{LF}$ induces the type-checking relation for all theories with meta-theory $\mathit{LF}$.

\begin{example}[{\mmt}-Theories]Below we give an {\mmt} theory $\Forms$, which will serve as the meta-theory of several logics introduced in this paper. It introduces all symbols needed to declare logical connectives and inference rules of a logic.
The syntax and semantics of this theory are defined in terms of type theory, e.g., the logical framework LF \cite{lf}.

\begin{tabular}{@{\hspace{-.55cm}}l@{\hspace{1cm}}l}
\begin{minipage}[b]{7cm}
$\kity$, $\rightarrow$, and $\lmbd$ are untyped constants representing the primitives of type theory. $\kity$ represents the universe of all types, $\rightarrow$ constructs function types $\alpha\to\beta$, and $\lmbd$ represents the $\lambda$-binder.
$\form$ is the type of logical formulas and  
%a typed constant representing the set of logical formulas. 
$\ded$ is a constant that assigns to each logical formula $F:\form$ the type $\ded\,F$ of its proof. 
\end{minipage} &
\begin{minipage}[b]{4cm}
\begin{twelfsig}
\tsig{\Forms}\\
$\kity$\\
$\rightarrow$\\
$\lmbd$\\
\decl{\form}{\kity}\\
\decl{\ded}{\form\to\kity}\\
\tsigend
\end{twelfsig}
\end{minipage}
\end{tabular}
\end{example}


%\begin{wrapfigure}{r}{4cm}
%\vspace{-3em}
%\begin{center}
%\begin{twelfsig}
%\tsig{\Forms}\\
%$\kity$\\
%$\rightarrow$\\
%$\lmbd$\\
%\decl{\form}{\kity}\\
%\decl{\ded}{\form\to\kity}\\
%\tsigend
%\end{twelfsig}
%\end{center}
%\vspace{-5em}
%\caption{An Example {\mmt} Theory}\label{fig:ex-mmt}
%\end{wrapfigure}

%{\mmt} theories consist of \emph{symbol declarations}. 
%{\mmt} views consist of \emph{symbol assignments}.  
%\emph{Constants} represent declarations of the base language.
%\emph{Structures} represent inheritance between theories.  
%\emph{Constant assignments} and \emph{structure assignments} represent assignments between constants and structures respectively. \emph{Links} are an {\mmt} module level notion that generalizes both structures and views. 
%Symbol declarations and assignments form a {\mmt} symbol level.  
%\emph{Terms} appear inside {\mmt} constants (e.g. as type or definition) and their grammar is motivated by the OpenMath grammar. They form the {\mmt} object level.


%%% Local Variables: 
%%% mode: latex
%%% TeX-master: "paper"
%%% End: 

% LocalWords:  mmt.tex wrapfigure vspace tikzpicture yscale omdoc emph Rudnicki
% LocalWords:  isabelle-isar aomp92 ednote RabKoh mmt-ontology overline hspace
% LocalWords:  rightarrow rightarrow ysep theo-name symb-name 5.2truecm xscale
% LocalWords:  cn mmt defemph frabe ptos pvs mathit hookrightarrow lf medskip
% LocalWords:  bnf hline bnfas lfkw cdot bnfalts stackrel ldots kity lmbd kity
% LocalWords:  ded ded twelfsig tsig tsigend
