\svnInfo $Id: csyntax.tex 9195 2012-04-20 12:26:50Z frabe $
\svnKeyword $HeadURL: https://svn.omdoc.org/repos/omdoc/projects/omdoc-2.0/pragmatic-strict/csyntax.tex $

\section{Pragmatic {\omdoc} Syntax}\label{sec:omdoc-pragmatic}

For implicit definitions we have the notation definition in listing~\ref{lst:notation}
(slightly simplified elemental XML notation). Recall from~\cite{KMR:presentation:08} that notation
definitions are essentially transformation rules that use named \lstinline|expr| elements
as variables in the heads and \lstinline|render| for the recursive calls of the
transformation.
\begin{lstlisting}[mathescape,label=lst:notation,morekeywords={pragmatic},
  caption=A Notation Definition for implicit Definitions]
<notation format="OMDoc-pragmatic">
  <prototype>
    <pragmatic name="$c$" extension="$\mmturi{\llquote{V}}\Description\implDef$">
      <expr name="prop"/>
      <expr name="proof"/>
    </pragmatic>
  </prototype>
  <rendering> 
   <definition type="implicit" name="$c$" exunique="$c.\exu$">
      <render name="prop"/>   
    </definition>
    <assertion name="$c.\exu$" type="theorem">
       <OMA>
          <OMS cd="$\Description$" name="\implDef"/>
          <render name="prop"/>
        </OMA>
    </assertion>
    <proof for="$c.\exu$"><render name="proof"/></proof>
  <rendering>
</notation>
\end{lstlisting}
Note that the ``direction'' of the notation definition may be confusing at first. But it
becomes natural: Here the \omdoc1.2 syntax can be regarded as a surface language into
which the \lstinline{statement} elements are presented into. But note that the notation
definitions can also be ``turned around'' for parsing. We are currently working on parsing
technology that integrates this idea.

\section{Pragmatic Concrete Syntax in \protect\sTeX}\label{sec:stex-pragmatic}
\lstset{language={[sTeX]TeX},morekeywords={extkey,extension}}

Now we show how to extend one of the {\omdoc} surface languages (formats intended for
human authoring and transformation to OMDoc) with pragmatic concrete syntax. We have
extended the \sTeX format~\cite{Kohlhase:ulsmf08} with an extension language, which we
present here as an example how the ideas presented in this paper can be extended to
semi-formal MKM formats. The next listing shows the extensions for implicit definitions.
The \lstinline|\extension| macro takes the name of the extension as a first required
argument and then uses the second and third arguments to define an environment of that
name in the usual way. The main problem in {\TeX}-based formats is argument passing; in
our extension language we use two ways. ``Small'' arguments are passed as key/value pairs;
we use the \lstinline|\extkey| macro to declare keys. ``Large'' arguments are naturally
packaged in environments of their own which are declared as ``continuations'' by the
\lstinline|cont| key in the optional argument of \lstinline|extension|. 

\begin{lstlisting}[caption=Extending \protect\sTeX by Implicit Definitions]
\begin{module}[id=ImplicitDefinitions]
\ldots
\extkey{idef}{name}
\extkey{idef}{title}
\extension{idef}
{\begin{def}[\namarg{idef}{title}]\label{idef.\namarg{idef}{name}}}
{\end{def}}
\extkey{exu}{for}
\extension[cont=idef]{exu}
{\begin{thm}[Existence and Uniqueness for \ref{\namarg{exu}{for}}]
\label{\namarg{exu}{for}.exu}}
{\end{thm}}
\extkey{exup}{for}
\extension[cont=idef]{exup}
{\begin{proof} (for \ref{\namarg{exu}})}
{\end{proof}}
\end{module}
\end{lstlisting}
Note the structural similarity of these extension declarations to the
\lstinline[language={[1.3]OMDoc}]|rendering| element in
Listing~\ref{lst:notation}.
Given this we can use the three induced environments
once we have imported the module above. 

\begin{lstlisting}[language={[sTeX]TeX},morekeywords={extkey,extension}]
\importmodule{ImplicitDefinitions}
\begin{idef}[name=exp,title=Exponential Function]
  The exponential function is that function $f$ with $f'=f$ and $f(0)=1$.
\end{idef}
\begin{exu}[for=exp]
  There is exactly one function $f$ with with $f'=f$ and $f(0)=1$.
\end{exu}
\begin{exup}[for=exp]
  \ldots
\end{exup}
\end{lstlisting}
Note that the OMDoc generated from the \sTeX can either be in the pragmatic concrete syntax from
Section~\ref{sec:omdoc-pragmatic} or directly in the concrete syntax from
Section~\ref{sec:omdoc-concrete}.


%%% Local Variables: 
%%% mode: latex
%%% TeX-master: "paper"
%%% End: 

% LocalWords:  csyntax.tex omdoc omdoc-pragmatic lst expr lstlisting mathescape
% LocalWords:  morekeywords mmturi llquote implDef exunique exu stex-pragmatic
% LocalWords:  lstset extkey ulsmf08 cont ldots idef namarg namarg thm exup exp
