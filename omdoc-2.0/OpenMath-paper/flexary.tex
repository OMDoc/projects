\svnInfo $Id: flexary.tex 6541 2007-06-21 17:46:22Z kohlhase $
\svnKeyword $HeadURL: https://svn.omdoc.org/repos/omdoc/projects/omdoc-2.0/OpenMath-paper/flexary.tex $
\section{A Mixfix Presentation Model for Flexary Terms}\label{sec:flexary}
The {\isabelle} presentation process allows to model all of the symbol characteristics
discussed in section~\ref{sec:characteristics} by a single notation definition: the
{\defemph{mixfix declaration}}. For an $n$-ary symbol, $f$ is a declaration of the form
$w_0\imarg{p_1}{\pi_1}w_1\imarg{p_2}{\pi_2}\ldots\imarg{p_n}{\pi_n}w_n:p$, where the $w_i$
are strings in the output language and $p$ is the {\defemph{output precedence}} of $f$. We
call the boxes {\defemph{argument specifications}}, the $p_i$ are {\defemph{input
    precedences}}. In accordance with Prolog, which, in contrast to {\isabelle},
predefines precedences for many common mathematical operators, and staying compatible with
previous {\omdoc} standards, we assign low precedence values to strongly binding operators
and high precedence values to weakly binding operators.

In the presentation process, a term $t$ is presented as a string
$\pres{t}{q}$, given an initial precedence $q$, according to the following definition:
\[\pres{f(t_1,\ldots,t_n)}{q}=w_0\pres{t_{\pi_1}}{p_1}w_1\ldots w_{n-1}\pres{t_{\pi_n}}{p_n}w_n\]
Note that the $w_i$ contain meta-tokens for brackets (actually pretty-printing blocks that
also handle indentation in {\isabelle}) that will be elided according to the ``current
precedence'' at that point.

A declaration of a right-associative infix operator $\to$ with precedence $p$ can be
modeled by a mixfix declaration $\imarg{p-1}{1}\to\imarg{p}{2}:p$. That means, the first
argument must be an atom or an operator that binds stronger than $\to$, or is bracketed
otherwise, and the second argument must have a maximum output precedence of $p$, which
would present an input of $a\to (b\to c)$ as $a\to b\to c$.

However, the {\isabelle} model is not directly applicable to presentation of {\openmath}
objects, as {\isabelle} crucially depends on the assumption that terms have a fixed arity.
In {\openmath} objects, applications can have flexible arities depending on the operator,
i.\,e.\ application nodes can have different numbers of children, even if they coincide in
the first child (the function). This is used to model a {\defemph{flexary}} addition
function, i.\,e.\ a function that takes any number of arguments, e.\,g.\ $@(add,1,2,3,4)$.
In {\isabelle}, we would have to model addition as a binary, associative operator, and the
term above as $plus(plus(1,2),plus(3,4))$ which would have the same presentation
$1+2+3+4$, but a different structure. Similarly, {\isabelle} only supports unary binding
constructions.

We will now extend the presentation model to the flexary case, and the types of
{\openmath} objects not covered by {\isabelle}, and come back to flexible elision and
abbreviation of non-brackets (we will come back to this in section~\ref{sec:elision}).
  
The first part of the flexary extension is to generalize the idea of an argument vector,
to a list of {\defemph{components}} of an {\openmath} object $O$ as follows (where indices
starts at $0$):
\begin{itemize}
\item for applications and errors: the list of children of $O$, e.\,g.,
  $(f,a_1,\ldots,a_n)$ for the object $@(f,a_1,\ldots,a_n)$,
\item for bindings: the binder, followed by the list of variables, followed by the body,
  e.\,g., $(b,x_1,\ldots,x_n,a)$ for the binding $\beta(b;x_1,\ldots,x_n;a)$,
\item for attributions: a three-element list consisting of the key and the value of the
  first attribution and the original object with the first attribution removed, e.\,g.,
  $(k_1,v_1,\alpha(k_2\mapsto v_2,\ldots,k_n\mapsto v_n;a))$ for $\alpha(k_1\mapsto
  v_1,\ldots,k_n\mapsto v_n;a)$,
\item otherwise: the object itself as a one-element list.
\end{itemize}
As a running example, consider the space of $n$-ary functions. We can model the function
space constructor as a flexary function symbol ${\sf{nfuns}}$ whose last argument is the
codomain and all previous ones the domains. We would like to present an {\openmath} object
$O\colon=@({\sf{nfuns}},A_1,\ldots,A_n,C)$ as $A_1\times \cdots \times A_n\to C$. By the
definition above, the set of components of the object $O$ is the list
${\sf{nfuns}},A_1,\ldots,A_n,C$.

In an obvious generalization of the mixfix notations, we would like to represent the
notation definition for $\sf{nfuns}$ as
$\fimarg{\times:\recu{p-1}}{1}{n-1}\to\imarg{\recu{p}}{n}:p$. In this, we had to
generalize the argument specification for the domains so it can deal with an argument
list. The new {\defemph{argument mapping specification}} represented by the box here
specifies the range of components it covers (the first $n-1$ arguments in our example),
their input precedence\footnote{Note that the input precedences of all members of the
  component range is equal, since they are indistinguishable as list members.}, and
finally, the {\defemph{separator}}, (the $\times$ operator in our example). The
precedences are chosen so that the arrows associate to the right in our example.
  
The general form a of a component mapping specification is $\fimarg{sep:N}be$, where $sep$
is the separator, $N$ is an arbitrary notation that may contain $\recu{p}$ to recurse into
the arguments with input precedence $p$, and $b/e$ are the {\defemph{component sequence
    begin/end indices}}.

Now let $w_0\fimarg{s_1:\recu{p_1}}{b_1}{e_1}w_1\ldots
w_{n-1}\fimarg{s_n:\recu{p_n}}{b_n}{e_n}w_n:p$ be a notation declaration for a function
$f$, then

  \[\pres{@(f,t_1,\ldots,t_m)}{q}=
        w_0\pres{t_{b_1}}{p_1}s_1\ldots s_1\pres{t_{e_1}}{p_1}w_1\ldots
        w_{n-1}\pres{t_{b_n}}{p_n}s_n\ldots s_n\pres{t_{e_n}}{p_n}w_n
  \]
  Note that an {\isabelle}-style argument specification $\imarg{p}\pi$ can be regained as
  $\fimarg{\epsilon:\recu{p}}\pi\pi$, where $\epsilon$ is e.\,g.\  the empty
  string. This legitimizes our example above.

  Note that again, the $w_i$ can contain brackets and other material that can be elided
  taking precedence and other factors into account. We will describe the necessary
  infrastructure at a conceptual level in the next section. 


%%% Local Variables: 
%%% mode: stex
%%% TeX-master: "presel"
%%% fill-column: 90
%%% sentence-end-double-space: nil
%%% End: 

% LocalWords:  stex presel
