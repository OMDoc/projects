\svnInfo $Id: compatibility.tex 6837 2007-09-23 00:07:39Z frabe $
\svnKeyword $HeadURL: https://svn.omdoc.org/repos/omdoc/projects/omdoc-2.0/OpenMath-paper/compatibility.tex $
\section{Compatibility}\label{sec:templates}
Here we show that the proposed infrastructure for notation definitions is a superset of
the functionality of other systems. We first show that we can re-gain the {\omdocv{1.2}}
syntax by specifying two new elements that get their meaning as abbreviations of the new
functionality. In a second step we develop a template-based variant of our presentation
technique to be compatible to (a functional superset of) the {\activemath} notation
definition system. 

\subsection{Compatibility to {\omdocv{1.2}}}

The proposed presentation syntax has one disadvantage: Information about the fixity or
associativity of a symbol is not marked up explicitly. Therefore, we specify
{\element{use}} elements in the next section which are very similar to the corresponding
{\omdocv{1.2}} syntax.

The {\element{style}}- and {\element{xslt}}-based infrastructure for notation definitions
in {\omdocv{1.2}} will not be supported. They have rightfully been described as
non-declarative, therefore ill-suited as a definitional mechanism and have tied the
presentation procedure to the XSLT programming language. The extended expressivity of the
proposed notation definition format seems to cover all the cases, where these two have
been needed in practice. We have re-formulated\ednote{MK: complete this} all the notation
definitions in the {\openmath} standard content dictionaries --- which cover the K-14
fragment of mathematics --- and have not found need for a procedural fall-back solution.


\subsection{Direct Specification of Symbol Characteristics}

We define a presentation element {\element{use}} which like its counterpart in
{\omdocv{1.2}} is based on the direct specification of symbol characteristics. But here we
define its meaning via abbreviations for certain presentations in the above syntax.
\begin{center}\small
\begin{tabular}{|l|l|p{3.2cm}|p{3.5cm}|}\hline
  Element  &  \multicolumn{2}{c|}{Attributes}  &  Content \\\cline{2-3}
  & Required & Optional & \\\hline\hline
  use           &          & egroup, elevel, precedence, bracket-style, fixity
                     & rbrack? lbrack? operator? separator?\\\hline
  operator      &          & egroup, elevel             & \llquote{item}* \\\hline\hline
  lbrack        &          & egroup, elevel             & \llquote{item}* \\\hline
  rbrack        &          & egroup, elevel             & \llquote{item}* \\\hline
  \multicolumn{4}{|l|}{\llquote{item} $\hat=$ (text $|$ element $|$ map $|$ attribution $|$
recurse)}\\\hline
\end{tabular}
\end{center}

A {\element{use}} may have the attributes
\begin{itemize}
\item {\attribute{fixity}{use}} with values {\attval{prefix}{fixity}{use}},
  {\attval{postfix}{fixity}{use}}, {\attval{infixl}{fixity}{use}},
  {\attval{infixr}{fixity}{use}}, and {\attval{infixlr}{fixity}{use}},
\item If fixity is {\attval{prefix}{fixity}{use}} or {\attval{postfix}{fixity}{use}}:
  {\attribute{bracket-style}{use}} with values {\attval{math}{bracket-style}{use}} and
  {\attval{lisp}{bracket-style}{use}}\ednote{discuss that! --CL},
  % \item optionally, {\attribute{crossref-symbol}{use}} whose value must be a
  %   whitespace-separated list of {\attval{lbrack}{crossref-symbol}{use}},
  %   {\attval{rbrack}{crossref-symbol}{use}}, and
  %   {\attval{separator}{crossref-symbol}{use}} with default value being the empty
  %   string.
\item {\attribute{precedence}{use}}.
\end{itemize}
Its children are at most one each of {\element{lbrack}}, {\element{rbrack}},
{\element{operator}}, and {\element{separator}}, all with \llquote{item}* content.
\ednote{FR: This should be improved: It should be possible that a use element gives only, e.g., fixity and precedence with the rest being inherited from default presentations. The motivation is that fixity and precedence typically depend on the symbol but not on the output format, whereas the other attributes typically depend on the home theory and the output format.}
Such a presentation specification can be translated easily into the above syntax. For
prefix operators,

\begin{minipage}{.53\textwidth}
\begin{lstlisting}[mathescape,frame=tlrb,numbers=none]
<use fixity="prefix" bracket-style="math" precedence="$p$">
  <lbrack>$\llquote{l}$</lbrack>
  <rbrack>$\llquote{r}$</rbrack>
  <operator>$\llquote{op}$</operator>
  <separator>$\llquote{s}$</separator>
</use>
\end{lstlisting}
\end{minipage}
\quad abbreviates\quad
\begin{minipage}{.28\textwidth}
\begin{lstlisting}[mathescape,frame=tlrb,numbers=none]
$\llquote{op}$
$\llquote{l}$
<map begin="1" end="-1">
  <separator>$\llquote{s}$</separator>
  <recurse precedence="$p$"/>
</map>
$\llquote{r}$
\end{lstlisting}
\end{minipage}

And

\begin{minipage}{.53\textwidth}
\begin{lstlisting}[mathescape,frame=tlrb,numbers=none]
<use fixity="prefix" bracket-style="lisp" precedence="$p$">
  <lbrack>$\llquote{l}$</lbrack>
  <rbrack>$\llquote{r}$</rbrack>
  <operator>$\llquote{op}$</operator>
  <separator>$\llquote{s}$</separator>
</use>
\end{lstlisting}
\end{minipage}
\quad abbreviates\quad
\begin{minipage}{.28\textwidth}
\begin{lstlisting}[mathescape,frame=tlrb,numbers=none]
$\llquote{l}^b$
$\llquote{op}$
$\llquote{s}$
<map begin="1" end="-1">
  <separator>$\llquote{s}$</separator>
  <recurse precedence="$p$"/>
</map>
$\llquote{r}^b$
\end{lstlisting}
\end{minipage}
where the superscript $b$ denotes that the attribute \snippet{level="bracket"}
has been added to the elements $l$ and $r$. Note how in the mathematical bracket
style brackets are always displayed, whereas they can be elided in the Lisp
style\ednote{@FR: Not in general! For postfix (e.\,g.\ factorial) and infix (most
  ordinary operators), yes, but for prefix? --CL}. Postfix operators are treated
correspondingly.

If omitted {\element{lbrack}}, {\element{rbrack}}, and {\element{separator}} default to
the empty elements\ednote{I thought in OMDoc 1.2 they defaulted to ( and )? --CL}, and
{\element{operator}} defaults to \snippet{map begin="0"/>}. {\attribute{fixity}{use}}
defaults to {\attval{prefix}{fixity}{use}}, {\attribute{bracket-style}{use}} to
{\attval{math}{bracket-style}{use}}, and {\attribute{precedence}{use}} to
{\attval{0}{precedence}{use}}.

% Here, and in the following, $C$ abbreviates \snippet{crossref="true"} if the
% corresponding element is in the value of \attribute{crossref-symbol}{use}, and is empty
% otherwise.

Infix operators require some care: An infix operator which is applied to $n$ arguments is
inserted $n-1$ times between the arguments. {\attval{infixlr}{fixity}{use}} is used for
associative operators for which no inner brackets are needed.
{\attval{infixl}{fixity}{use}} and {\attval{infixr}{fixity}{use}} refer to left and
right-associative operators, respectively. In none of these cases do we generate inner
brackets, not even elidable ones. This permits us to distinguish, e.\,g., the $n$-ary
application of addition from successive application of binary addition.

Elidable or unelidable brackets are only generated when the several applications of the
operator are nested. This is achieved by different input precedences when
expanding. Associative, right-associative, and left-associative infix specifications
expand to, respectively,
\begin{center}
\begin{minipage}{.3\textwidth}
\begin{lstlisting}[mathescape,frame=tlrb,numbers=none]
$\llquote{l}^b$
<map begin="1" end="-1">
  <separator>$\llquote{op}$</separator>
  <recurse precedence="$p$"/>
</map>
$\llquote{r}^b$
\end{lstlisting}
\end{minipage}
\quad
\begin{minipage}{.3\textwidth}
\begin{lstlisting}[mathescape,frame=tlrb,numbers=none]
$\llquote{l}^b$
<map begin="1">
  <separator/>
  <recurse precedence="$p-1$"/>
</map>
<map begin="2" end="-1">
  <separator/>
  $\llquote{op}$
  <recurse precedence="$p$"/>
</map>
$\llquote{r}^b$
\end{lstlisting}
\end{minipage}
\quad
\begin{minipage}{.3\textwidth}
\begin{lstlisting}[mathescape,frame=tlrb,numbers=none]
$\llquote{l}^b$
<map begin="1" end="-2">
  <separator/>
  <recurse precedence="$p$"/>
  $\llquote{op}$
</map>
<map begin="-1">
  <separator/>
  <recurse precedence="$p-1$"/>
</map>
$\llquote{r}^b$
\end{lstlisting}
\end{minipage}
\end{center}

\subsection{Compatibility to {\activemath}}

  The {\activemath} system~\cite{URL:activemath} uses a pattern-based approach that
  synthesizes ideas from the {\omdocv{1.0}} infrastructure and~\cite{Naylor:conversion}
  for XSLT template generation.
\begin{lstlisting} 
<symbolpresentation xmlns="..." id="comb1bk" xref=".../bk">
  <notation format="html|pmathml|TeX" language="de">
    <math xmlns="http://www.w3.org/1998/Math/MathML">
      <mrow><mo>(</mo><mfrac linethickness="0"><mi>n</mi><mi>k</mi></mfrac><mo>)</mo></mrow> 
    </math>
    <OMOBJ xmlns="http://www.openmath.org/OpenMath">
      <OMA><OMS cd="comb1" name="bk"/><OMV name="n"><OMV name="k"></OMA>
    </OMOBJ>
  </notation>
</symbolpresentation>
\end{lstlisting}
  The presentation approach outlined in~\cite{ManLib:apo05} directly uses the names of
  {\openmath} variables and the content of {\mathml} {\element{mi}} elements for the
  necessary cross-referencing between formats, confusing the role of meta-variables and
  object variables. Conceptually, this is a step backwards from~\cite{Naylor:conversion}
  (where the distinction was somewhat tentative) and makes it impossible to exclude
  variables from the cross-referencing (and therefore recursion) of the presentation
  procedure. The main contribution here is the provision of the
  {\textit{jEditOQMath}}~\cite{jeditoqmath:web} developed by P.\
  Libbrecht~\cite{Lib:authoring} with a {\openmath} Presentation Editor (OMPE)
  plug-in~\cite{oqmath:web} for the visual authoring of {\element{symbolpresentation}}
  elements.

\begin{newpart}{MK: we have to talk about this}
  \subsection{A Pattern-Based Approach to Flexary Mixfix Notations}

%\begin{newpart}{FR: pattern example for mathml draft}
%\documentclass{article}
\usepackage{listings,ed}
\usepackage{lstomdoc}

\begin{document}

Idea of semantics: A substitution component is given by arglist/.../arglist/arg and accessed by iterate/.../iterate/render with corresponding names. The substitution component at the current position in the match of arglist/.../arglist/arglist is accessed by iterate/.../iterate/render, too. If a render cannot be resolved, the name of its iterate parent is prepended.

Flexary bounded integral as non-atomic binder presented by using nested patterns

\ednote{I use Latex because I don't know MathML.}
\ednote{I don't use the official symbol and theory names because I don't know them.}

\begin{lstlisting}
<presentation format="latex">
  <prototype>
    <OMBIND>
      <OMA>
        <OMS name="int"/>
        <arglist name="domains">
          <OMA>
            <OMS name="interval"/>
            <arg name="lower_bound"/>
            <arg name="upper_bound"/>
          </OMA>
        </arglist>
      </OMA>
      <OMBVAR>
        <arglist name="variables"/>
      </OMBVAR>
      <arg name="condition"/>
      <arg name="body"/>
    </OMBIND>
  </prototype>
  <rendering>
    <text>\mathop{</text>
    <iterate name="domains">
      <separator><text> </text></separator>
      <text>\int_{</text>
      <render name="lower_bound"/>
      <text>}^{</text>
      <render name="upper_bound"/>
      <text>}</text>
    </iterate>
    <text>}_{</text>
    <render name="condition"/>
    <text>}</text>
    <render name="body"/>
    <iterate name="variables" reverse="Yes">
      <text>d <text>
      <render name="variables"/>
    </iterate>
\end{lstlisting}

Intended presentation of \[\beta(@(int,interval(a_1,b_1),interval(a_2,b_2)),x_1,x_2,@(>,x_1,x_2),@(f,x_1,x_2))\] is
\[\mathop{\int_{a_1}^{b_1}\int_{a_2}^{b_2}}_{x_1>x_2}f(x_1,x_2)d x_2 d x_1\]

Note: This does not work for a flexary integral where set and interval bounds are mixed.

Note: This does not work if the bound variables have attributions. (The same problems occurs for the pattern that matches a lambda binder that is already used in the specification.)
\end{document}

%\end{newpart}


  Template-based approaches are often simpler to understand for authors, moreover, they
  are the only intuitive way to capture notations like $\sin^2(x)$ for the square of the
  value of the sine function at $x$. Here we propose to directly follow the lead of the
  Naylor\&Watt system described in Section~\ref{sec:notdef}, re-using as much of the
  infrastructure we have built up until now as possible. Furthermore, we will drop the
  distinction between a semantic template and the notation generators. We claim that
  template-based notation definitions should only contain information about
  ``corresponding'' representations, and that presentation generation is only one
  direction of transformation between representations (from machine-oriented to
  human-oriented ones). The the symmetricity of the ``correspondence relation'' is
  immediately perspicuous if we have templates for more than one machine-oriented format,
  e.\,g.\ {\openmath} and content {\mathml}. But parsing of human-oriented formats should in
  principle be able to use the information in the ``notation correpsondence'' in the
  opposite of the ``presentation direction''. We should be able to derive parsing-oriented
  grammars taking into account precedences and the extended elision specifications
  developed above. In this paper though, we will not make use of this generalization and
  only concentrate on the presentation direction. Concretely, we assume that the
  presentation algorithm either directly uses the {\openmath} template for pattern
  matching internally, or generates XPath expression for the
  {\attribute[ns-elt=xsl]{match}{template}} attribute in the {\element[ns-elt=xsl]{template}}
  element as in {\mylstref{template}}.


  Finally, we keep the set of corresponding templates loosely coupled, rather than enging
  them in a {\element{Notation}} element as in {\mylstref{lst:template-based}}: Two
  {\element{presentation}} elements are in correspondence, if they reference the same
  symbol in the {\attribute{for}{presentation}} element and have the same
  {\attribute{role}{presentation}} value.

  Consider for instance the case of the typing judgment from {\mylstref{typing-judgment}},
  here we would just add the following template. 
\begin{lstlisting}[mathescape]
<presentation format="OM" for="#typing-judgment">
  <OMA><OMS cd="types" name="typing-judgment"/>
    <arg name="context"/><arg name="sig"/><arg name="term"/><arg name="type"/>
  </OMA>
</presentation>
\end{lstlisting}
  Unfortunately, we cannot directly write this in our grammar (nor would we like allow
  this, since we want to support an open-ended set of XML-based formats). Therefore we
  write instead
\begin{lstlisting}[mathescape]
<presentation format="OM" role="application" for="#typing-judgment">
  <element name="OMA">
    <element name="OMS">
      <attribute name="cd"><text>types</text></attribute>
      <attribute name="name"><text>typing-judgment</text></attribute>
    </element>
    <arg name="context"/><arg name="sig"/><map name="term"/><map name="type"/>
  </element>
</presentation>
\end{lstlisting}
  This template is assembled using the {\omdoc} {\element{element}} and
  {\element{attribute}} elements, rather than the {\openmath} elements themselves. We
  consider this representation above to be equivalent to the one that uses the target
  elements literally.

  Following XSLT terminology, we call we speak of the {\defemph{native}} encoding for the
  former representation and the {\defemph{literal inclusion}} encoding for the
  latter. Note that the native encoding has great advantages for validation, while the
  literal inclusion encoding is simpler to read for humans. They are obviously identical
  in expressivity and trivially isomorphic. We therefore assume that native form is used
  for storage and a pre-process or a tool like OMPE will present equivalent literal
  inclusion syntax for editing and user interaction. In particular, we will write our
  examples in literal inclusion encoding.

  \begin{newpart}{@MK: I only understood this stuff when you gave the additional PMML
      pattern and told me that one way of the presentation could be ``pattern matching
      OM'' $\to$ ``output PMML''. Therefore, I want an PMML example here and an
      explanation; please adapt this draft. --CL}
\begin{lstlisting}[mathescape]
<presentation format="pmathml" for="#typing-judgment">
  <mrow>
    <arg name="context"/>
    <msub><mo>$\vdash$</mo><arg name="sig"/></msub>
    <arg name="term"/>
    <mo>:</mo>
    <arg name="type"/>
  </mrow>
</presentation>
\end{lstlisting}
    
    When for the same symbol there is a presentation pattern $p_{in}$ for the given input
    format (e.\,g.\ {\openmath}) and $p_{out}$ for the desired output format (e.\,g.\
    Presentation {\mathml}), the input would be matched against $p_{in}$, assigning the
    meta-variables named by the {\attribute{name}{map}} attribute. Then, the output would
    be rendered according to $p_{out}$, filling in the results of recursively rendering
    the values of the meta-variables at the indicated positions.
  \end{newpart}

  Note that the native encoding also clarifies the role of the {\element{map}} elements as
  meta-variables, and the contribution of the other presentation elements.

\begin{lstlisting}[mathescape]
<presentation for="#plus" role="constant" format="OM"><OMS cd="arith1" name="plus"/></presentation>
<presentation for="#plus" role="application" format="OM">
  <OMA><OMS cd="arith1" name="plus"/><map begin="1" end="-1" name="summands"/></OMA>
</presentation>
<presentation for="#plus" role="application" format="ascii latex">
  <element name="mo" egroup="lbrack"><text>(</text></element>
  <map name="summands"><separator><arg pos="0"/></separator><recurse/></map>
  <element name="mo" egroup="lbrack"><text>(</text></element>
</presentation>
\end{lstlisting}
  \ednote{MK: say something about the referencing with @name and @begin/end/pos, and where
    what is redundant. But @name is always a good documentation of the intended behavior.}
\end{newpart}




%%% Local Variables: 
%%% mode: stex
%%% TeX-master: "presel"
%%% fill-column: 90
%%% sentence-end-double-space: nil
%%% End: 


% LocalWords:  CL rbrack lbrack infixl infixr infixlr crossref whitespece tlrb
% LocalWords:  mathescape FR OMDoc assoc ary op ednote symbolpresentation xmlns
% LocalWords:  bk xref html pmathml de mrow mo mfrac linethickness mi OMOBJ OMA
% LocalWords:  OMS cd OMV MK jEditOQMath OMPE Vorher hieß es noch Das hier ist
% LocalWords:  doch nicht Warum denn auf einmal Ausgabeformat OM Im ersten bzw
% LocalWords:  Beispiel wenn wir von sprechen wollen wäre MathML sinnvoll aber
% LocalWords:  eine Identitäts Abbildung Außerdem geht ein der Eingabesprache
% LocalWords:  macht das und Ausgabesprache PMML nimmt gematchten Ausdrücke pre
% LocalWords:  crid arith stex presel egroup elevel
