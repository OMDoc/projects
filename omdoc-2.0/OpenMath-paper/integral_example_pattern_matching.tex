\documentclass{article}
\usepackage{listings,ed}
\usepackage{lstomdoc}

\begin{document}

Idea of semantics: A substitution component is given by arglist/.../arglist/arg and accessed by iterate/.../iterate/render with corresponding names. The substitution component at the current position in the match of arglist/.../arglist/arglist is accessed by iterate/.../iterate/render, too. If a render cannot be resolved, the name of its iterate parent is prepended.

Flexary bounded integral as non-atomic binder presented by using nested patterns

\ednote{I use Latex because I don't know MathML.}
\ednote{I don't use the official symbol and theory names because I don't know them.}

\begin{lstlisting}
<presentation format="latex">
  <prototype>
    <OMBIND>
      <OMA>
        <OMS name="int"/>
        <arglist name="domains">
          <OMA>
            <OMS name="interval"/>
            <arg name="lower_bound"/>
            <arg name="upper_bound"/>
          </OMA>
        </arglist>
      </OMA>
      <OMBVAR>
        <arglist name="variables"/>
      </OMBVAR>
      <arg name="condition"/>
      <arg name="body"/>
    </OMBIND>
  </prototype>
  <rendering>
    <text>\mathop{</text>
    <iterate name="domains">
      <separator><text> </text></separator>
      <text>\int_{</text>
      <render name="lower_bound"/>
      <text>}^{</text>
      <render name="upper_bound"/>
      <text>}</text>
    </iterate>
    <text>}_{</text>
    <render name="condition"/>
    <text>}</text>
    <render name="body"/>
    <iterate name="variables" reverse="Yes">
      <text>d <text>
      <render name="variables"/>
    </iterate>
\end{lstlisting}

Intended presentation of \[\beta(@(int,interval(a_1,b_1),interval(a_2,b_2)),x_1,x_2,@(>,x_1,x_2),@(f,x_1,x_2))\] is
\[\mathop{\int_{a_1}^{b_1}\int_{a_2}^{b_2}}_{x_1>x_2}f(x_1,x_2)d x_2 d x_1\]

Note: This does not work for a flexary integral where set and interval bounds are mixed.

Note: This does not work if the bound variables have attributions. (The same problems occurs for the pattern that matches a lambda binder that is already used in the specification.)
\end{document}
