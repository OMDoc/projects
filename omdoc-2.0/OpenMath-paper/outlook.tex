\svnInfo $Id: outlook.tex 6570 2007-06-27 08:40:11Z clange $
\svnKeyword $HeadURL: https://svn.omdoc.org/repos/omdoc/projects/omdoc-2.0/OpenMath-paper/outlook.tex $
\section{Conclusion and Outlook}\label{sec:outlook}

The advantage of content-oriented representation formats for mathematical formulae is that
they are independent or a specific output format, and that human-oriented presentations
can be generated taking into account user preferences, device constraints, etc. To realize
this potential, we need presentation algorithms that are
\begin{description}
\item[knowledge-based] The knowledge about notations and their use is an integral part of
  our mathematical expertise. If a presentation algorithm is not, then it becomes
  unmaintainable.
\item[extensible] Just as mathematics itself, mathematical notation is never finished; new
  notations are always invented, and they are introduced and discarded on the fly.
\item[adaptive] They should allow to take the user's preferences into account. In the end,
  the purpose of mathematical notation is to support a reader in understanding mathematics
  as effortlessly as possible --- and people differ in their cognitive styles.
\item[mathematical] They should be able to model as many mathematical practices as
  possible, so that mathematicians are not hindered by system restrictions in the
  available notations.
\item[efficient] Presentation generation should not bog down servers, and should run on
  all kinds of clients with little lag. The time of the working mathematician is precious!
\end{description}
The first three of these aspects call for an meta-mathematical language for
{\emph{notation definitions}} which can be managed by MKM tools. A corollary of the
manageability postulate is that the notation definition language should be
{\emph{declarative}}. 

In this paper, we have presented an infrastructure for declarative notation definitions
building on ideas from {\omdocv{1.2}} and the presentation system of the {\isabelle}
theorem prover. We have incorporated functionality for flexary applications and binders
necessary to apply these ideas to {\openmath} and content {\mathml}, and we have extended
the {\isabelle}'s precedence-based elision algorithm with general flexible elision
functionality, which covers most mathematical elision practices. We conjecture that
others, like Church's dot notation can be specified by allowing more values for the
{\attributeshort{egroup}} attribute.

Actually, the dot notation is a representative of a more general, user-adaptive practice
that we do not cover in this paper: {\emph{abbreviation}}. If a formula is too large or
complex to be digested in one go, mathematicians often help their audience by abbreviating
parts, which are explained in isolation or expanded when the general picture has been
grasped. We feel that the abbreviation generation problem shares many aspects with
elisions, but is less structured, and more dependent on modeling the abilities and
cognitive load of the reader. Therefore we expect the elision techniques presented in this
paper to constitute a first step into the right direction, but also that we need more
insights to solve the abbreviation generation problem.

Another problem we have not addressed in this paper even though it is very related is the
problem of reversing generation in the parsing process for mathematical formulae\ednote{I
  guess we did in the compatibility section. --CL}: In some situations, the symbol
characteristics are sufficient to allow a mathematical software system to reconstruct the
content structure from its presentation. In the case of the {\isabelle} system, this
property is essential to ensure unambiguous input. In this paper, we will not pursue the
question of parseability, since the flexible elision of arguments is not sufficiently
understood to warrant a restriction to an invertible presentation process. In fact, we
have argued elsewhere\cite{Kohlhase04:stex} that the process of parsing mathematical
formula is AI-hard\footnote{i.e. we cannot solve the problem short of succeeding at
  Artificial Intelligence} in the worst case, since it pre-supposes a mathematical
understanding of the communicated concepts.

We have implemented the approach presented in this paper in a Scala class\ednote{@CL/FR:
  cite, is that correct?} to evaluate the expressivity and efficiency of our approach.
First results are very promising; we will release the finished implementation under an
open source license as a basis for practical presentation systems.

After a thorough evaluation we want to incorporate the presentation infrastructure
presented in this paper into the {\omdocv{2.0}} format currently under development, and we
propose it as part of the content dictionary mechanism of the upcoming {\mathml}3
recommendation.

%%% Local Variables: 
%%% mode: stex
%%% TeX-master: "presel"
%%% fill-column: 90
%%% sentence-end-double-space: nil
%%% End: 

% LocalWords:  egroup Scala CL FR presel
