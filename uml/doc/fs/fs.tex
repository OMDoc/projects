\documentclass[pdftex,a4paper]{article}
\usepackage{url}
\usepackage{graphicx}
\usepackage{enumerate}
\usepackage{amsmath}
\usepackage{amssymb}
\usepackage{amsbsy}
\usepackage{latexsym}
\usepackage{bbding} % nice arrows
\usepackage{listings}
\usepackage{lstscala}
\usepackage{var,cmt}
\usepackage{acro}
\usepackage{xspace}
\usepackage{tikz}
\usetikzlibrary{shapes}
\usetikzlibrary{arrows}
\tikzstyle{default}=[font=\sffamily,>=triangle 60]
\tikzstyle concept=[font=\sffamily\bfseries,draw,minimum height=3.5ex,rounded corners]

%%%%%%%%%%%%%%%%%%%%%%%%%%%%%%%%%%%%%%%%%%%%%%%%%%%%%%%%%%%%%%%%%%%%%%%%%%%%%%%%%%%%%%%%%%
% SVN info
%\usepackage[eso-foot,today]{svninfo}
%\usepackage[final]{svninfo}
%\svnInfo $Id: fs.tex 1850 2008-06-20 16:28:21Z nmueller $
%\svnKeyword $HeadURL: https://svn.omdoc.org/uml/doc/fs.tex $
%
% Editor notes & variants
%\declarevariant{longversion}
%\declarevariant{shortversion}
%\setvariant{longversion}

\usepackage[show]{ed}
%\usepackage[hide]{ed}
%%%%%%%%%%%%%%%%%%%%%%%%%%%%%%%%%%%%%%%%%%%%%%%%%%%%%%%%%%%%%%%%%%%%%%%%%%%%%%%%%%%%%%%%%%

%\lstset{language=XML,basicstyle=\small,frame=tb, float=htb, columns=flexible}

\newcommand{\png}{\textsc{PNG}\xspace}
%\newcommand{\dtd}{\textsc{DTD}\xspace}
\newcommand{\rnc}{\textsc{RNC}\xspace}
\newcommand{\ocl}{\textsc{OCL}\xspace}
\newcommand{\argouml}{\textsc{ArgoUML}\xspace}
\newcommand{\pgml}{\textsc{PGML}\xspace}
%\newcommand{\uml}{\textsc{UML}\xspace}
\newcommand{\svg}{\textsc{SVG}\xspace}
%\newcommand{\omdoc}{\textsc{OMDoc}\xspace}
%\newcommand{\xhtml}{\textsc{XHTML}\xspace}
%\newcommand{\xslt}{\textsc{XSLT}\xspace}

\title{Functional specification for UMLProc}

\pagestyle{plain}

\begin{document}

\maketitle{}

The aim of this project is a basic integration of \uml into \omdoc. To achieve this, two main tasks
have to be completed:
\begin{itemize}
\item Specification and documentation of an \omdoc module for \uml,
\item Programming of an \uml processor that converts \omdoc documents with \uml diagrams to
    \xhtml.
\end{itemize}

\section{Specification of an \omdoc / \uml module}
The \uml module for \omdoc should specify how to represent certain \uml diagrams in \omdoc.
Relevant diagram types have to be identified and may include, but are not restricted to Class
Diagrams, Sequence Diagrams, Use Case Diagrams, Collaboration Diagrams, State Diagrams and
Deployment Diagrams \ednote{These are the diagram types supported by {\argouml}}. Semantic
associations between these diagrams and native \omdoc elements need to be identified and
documented. This comprises the question of how to include \uml diagrams in \omdoc theories and how
to include \omdoc statements and objects in \uml diagrams.

\xmi will serve as a basis for the specification of an \xml representation of \uml models, but
solely for the content of these diagrams. As the \xml representation of the graphical view on these
\uml models has not been sufficiently standardized and is mainly tool-dependent, it is sufficient
to choose such an \xml representation on the basis of the choice of the \uml modeling tool. As it
is intended to modify the modeling tool, an open source modeling tool is most convenient for this
project, hence \argouml seems to be the best choice. \argouml uses the Precision Graphics Markup
Language (\pgml) for representing the graphical representation of \uml diagrams.

Fig.~\ref{fig:example:input} shows the basic structure of a \omdoc document with mixed-in \uml
elements. The \xmlelem{uml:model} element contains content markup for \uml models and formulae
written in \omdoc. Such formulae might describe constraints, pre- and postconditions as it is
usually done with the Object Constraint Language (\ocl) in the \uml community. A complete example
is provided with the file \textit{input-v0.omdoc}.

\begin{myfig}{example:input}{A sample \omdoc document with \uml}
\begin{lstlisting}[language=XML,mathescape,breaklines=true]
<omdoc>
 <theory>
  <omtext>
   $\ldots$
  </omtext>
  <uml>
   <uml:model>
    $\ldots$
   <omobj>
    $\ldots$
   </omobj>
   </uml:model>
   <pgml>
    $\ldots$
   </pgml>
  </uml>
 </theory>
</omdoc>
\end{lstlisting}
\end{myfig}

This first task can be split up into three subtasks that will be carried out in parallel:
\begin{itemize}
\item Converting the existing \dtd for \xmi to \rnc
\item Extending the existing \rnc for \omdoc to possibly include \uml diagrams where
    appropriate
\item Documenting the newly introduced elements and the relations to existing \omdoc elements
\end{itemize}

The preliminary design goals are:
\begin{itemize}
\item Support a subset of the \uml diagrams that is sufficient for the whole software design
    process
\item Provide at least as much expressiveness as \ocl
\end{itemize}

\section{Presentation of \omdoc/\uml documents}
To enable a basic workflow for \omdoc / \uml documents it is necessary to provide means of viewing
these documents in a webbrowser. The current conversion process of \omdoc documents to \xhtml
documents needs to be extended to support conversion of \uml models. More concretely, a conversion
from \pgml to \svg needs to be specified, as \pgml is not supported by recent webbrowser, but \svg
is.

It is intended to solve this task by writing an \xslt that converts the subset of \pgml that is
used in \argouml to \svg. In a second step the question of how to incorporate the mathematical
formulae into the diagrams has to be solved. Possible solutions are:
\begin{itemize}
\item Extend \argouml to automatically add the formulae in the \pgml element and extend the
    \xslt accordingly
\item Display the formulae separately using the existing conversion tools while linking them to
    the appropriate elements in the \uml diagram
\end{itemize}


\end{document}
