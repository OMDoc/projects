\documentclass[pdftex,a4paper]{article}

\title{Phonebook example: Relations}
\author{S{\"o}nke Holsten\\[1ex]
\begin{minipage}{.5\textwidth}
  \centering
  Jacobs University Bremen \\
  D-28759, Bremen, Germany\\
  s.holsten@jacobs-university.de
\end{minipage}}

\pagestyle{plain}

\begin{document}

\maketitle{}

\begin{tabular}{|l|l|l|}
\hline
ElementID 1 & ElementID 2 & Relation \\
\hline
Phonebook-Requirements & uml-phonebook & satisfiedBy \\ \hline
func1-req & uml-phonebook-op1 & satisfiedBy \\ \hline
func2-req & uml-phonebook-op2 & satisfiedBy \\ \hline
func3-req & uml-phonebook-op3 & satisfiedBy \\ \hline
func1-name & uml-phonebook-op1 & nameEquality \\ \hline
func2-name & uml-phonebook-op2 & nameEquality \\ \hline
func3-name & uml-phonebook-op3 & nameEquality \\ \hline
nfunc1-req-text & nfunc1-req-om & formalizes \\ \hline
nfunc1-req-om & uml-phonebook & isInvariantOn \\ \hline
nfunc1-req-om & uml-phonebook-op1 & isPostconditionOf \\ \hline
nfunc1-req-om-e1 & uml-phonebook-entry & isOfType \\ \hline
nfunc1-req-om-e2 & uml-phonebook-entry & isOfType \\ \hline
nfunc1-req-om-PB & uml-phonebook & isOfType \\ \hline
nfunc1-req-om-e1.phone & uml-phonebook-entry-number & typeEquality \\ \hline
nfunc1-req-om-e2.phone & uml-phonebook-entry-number & typeEquality \\ \hline
nfunc1-req-om-e1.phone & uml-dt-int & isOfType \\ \hline
nfunc1-req-om-e2.phone & uml-dt-int & isOfType \\ \hline
pgml-phonebook & uml-phonebook & visualizes \\ \hline
\end{tabular}

\begin{itemize}
\item The \textit{satisfiedBy}-relation holds between a functional requirement and the
    function(s) that provide this functionality.
\item The \textit{nameEquality}-relation holds between elements that are named the same.
\item The \textit{formalizes}-relation holds between two elements that represent the same at
    different levels of formalization.
\item The \textit{isInvariantOn}-relation holds between a openmath formula and a uml class
    element, where the formula is an invariant on the uml class
\item The \textit{isPostconditionOf}-relation holds between a openmath formula and a uml
    operation element, where the formula is a postcondition of the uml operation
\item The \textit{isOfType}-relation holds between an element and another element that defines
    the type of the first element, i.e. a uml class element or a uml datatype element
\item The \textit{typeEquality}-relation holds between two elements that have the same type.
\item The \textit{visualizes}-relation holds between a uml element and a pgml element that
    visualizes an aspect of the uml element
\end{itemize}

\end{document}
