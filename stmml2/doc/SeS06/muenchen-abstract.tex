Title: Capturing the Content of Physics

Abstract:

Today's scientific documents are {\emph{machine-readable}}, therefore we can publish them
on the web, send them in e-mails, and search for words in them via Google. However, we
cannot search for a relevant experiment, check dimensions in equations, or change units or
coordinate systems in an exposition. For this we would have to make the documents also
{\emph{machine-understandable}} by capturing the content of the embedded knowledge.

To facilitate this, we propose to realize a content markup language PhysML by extending
the OMDoc format (Open Mathematical Documents) was initially developed as a content-markup
format for mathematical documents by an infrastructure for physics, concentrating on
{\emph{observables}}, {\emph{systems}}, and {\emph{experiments}}. The semantic information
embedded in OMDoc documents has for instance been used by eLearning systems to automate
user-adaption of course materials or for semantic search for mathematical formulae. OMDoc
marks up knowledge on three levels:
\begin{description}
\item[Object Level] it uses OpenMath and content MathML for objects represented as
  mathematical formulae;
\item[Statement Level] OMDoc provides original markup primitives that allow to specify the
  semantical structure and interdependencies of theorems, axioms, definitions, proofs, and
\item[Context-Level] statements are grouped into mathematical theories, whose structure
  can be expressed by a rich set of theory morphisms.
\end{description}
Our extension only changes the statement level; the object and context levels stay the
same: they model the general ``scientific method''. Thus the extended three-level approach
to knowledge representation can be used as an open basis for true eScience.

% LocalWords:  Google PhysML
